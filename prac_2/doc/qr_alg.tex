\section{Transformacja Householdera}

\subsection{Podstawy teoretyczne}

Transformacja Householdera [ \worldflag[length=12px, width=7px]{GB}: Householder transformation ] to liniowe przekształcenie poprzez odbicie punktu wokół płaszczyzny, lub hiperpłaszczyzny, która zawiera początek układu współrzędnych. Płaszczyzna wokół której obracamy jest zdefiniowana przez jednostkowy wektor $u$ do niej normalny, a więc odbicie względem niej to $x$ pomniejszony o dwa rzuty na $u$:
$$x'=x-2u\langle x,u\rangle=x-2u(u^*x)$$
co dla przestrzeni rzeczywistej wynosi
$$x'=x-2u(u^Tx).$$
Macierz tego odbicia to
$$P=I-2uu^*$$
i jest ona Hermitowska:
$$P^*=(I-2uu^*)^*=I^*-(2uu^*)^*=I-2(uu^*)^*=I-2(u^*)^*u^*=I-2uu^*=P$$
oraz unitarna (czyli $P^*P=PP=I$):
\begin{align*}
    P^*P&=P^2=(I-2uu^*)^2=I-4uu^*+4(uu^*)^2=\\
    &=I-4uu^*+4u(u^*u)u^*=I-4uu^*+4u\langle u,u\rangle u^*=\\
    &=I-4uu^*+4uu^*=I
\end{align*}
a więc w przypadku rzeczywistym dostajemy macierz symetryczną i ortogonalną, czyli taką jakiej szukamy.

Niech teraz $A$ będzie macierzą $m\times m$, której formę $QR$ chcemy znaleźć, a $a_1,...,a_m$ będą wektorami odpowiadającymi jej kolumnom. Dalej, niech $e_1,...,e_m$ będą wektorami ze standardowej bazy przestrzeni $\R^m$ i ustalmy
$$v=a_1-\|a_1\|e_1$$
$$u={v\over\|v\|}.$$
Wektor $u$ jest jednostkowym wektorem pewnej płaszczyzny przechodzącej przez początek układu współrzędnych, możemy więc dla niego znaleźć macierz Householdera
$$P'_1=I-2uu^*.$$
Zauważmy, że
$$P'_1a_1=\begin{pmatrix}\|a_1\|\\0\\0\\...\\0\end{pmatrix}$$
czyli zaczynamy tworzyć macierz górnotrójkątną. Proces transformacji Householdera możemy powtórzyć dla macierzy $P_1A$ bez pierwszej kolumny i wiersza, co da nam macierz $P'_2$ która dla $P'_2a_2$ daje wektor niezerowy tylko na pierwszej współrzędnej. Jednak $P'_2$ jest $(m-1)\times(m-1)$, więc musimy ją rozciągnąć, chociażby dodając identyczność w lewym górnym rogu. Rozciągając tę procedurę na przypadek ogólny, mamy
$$P_k=\begin{pmatrix}
    I_{k-1}&*\\
    0&P'_k
\end{pmatrix}$$
gdzie $I_{k-1}$ to identyczność ale na $\R^{k-1}$. 

Szukana przez nas macierz górnotrójkątna ma zatem postać
$$R=P_m...P_1A$$
natomiast szukana macierz ortogonalna to
$$Q=(P_m...P_1)^{-1}=P_1^{-1}...P_m^{-1}=P_1...P_m$$
z faktu, że każda z macierzy $P_k$ jest unitarna i hermitowska.

Zakładamy, że macierz jest podawana w liście wektorów zawierających kolejne jej kolumny:
\begin{lstlisting}{language=ps}
    function decomposition (A)
        R = A
        for i in 1:m
            normA = norm(A[i])
            s = -sign(A[i][i])
            v = A[i] - s * normA
            u = v / norm(v)

            nie wiem, sam pisz to R=P_iR

\end{lstlisting}

\subsection{Wyniki}
