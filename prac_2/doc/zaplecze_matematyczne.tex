\section{Rozwiązywanie równań liniowych}

Mając dany układ równań liniowych:
$$
\begin{cases}
    a_{11}x_1+a_{12}x_2+...+a_{1n}x_n=b_1\\
    a_{21}x_1+a_{22}x_2+...+a_{2n}x_n=b_2\\
    ...\\
    a_{n1}x_1+a_{n2}x_2+...+a_{nn}x_n=b_n
\end{cases}
$$
możemy go opisać w postaci macierzy. Macierz główna tego układu równań to macierz zawierająca wszystkie współczynniki przy zmiennych [$X$]:
$$A=\begin{bmatrix}
    a_{11}&a_{12}&...&a_{1n}\\
    a_{21}&a_{22}&...&a_{2n}\\
    ...&...&...&...\\
    a_{n1}&a_{n2}&...&a_{nn}
\end{bmatrix}.$$
Jeśli do macierzy głównej dołączymy wektor zawierający wszystkie wyrazy wolne [$B$], to dostaniemy macierz rozszerzoną tego układu:
$$
A|B=\begin{bmatrix}\begin{array}{cccc|c}
    a_{11}&a_{12}&...&a_{1n}&b_1\\
    a_{21}&a_{22}&...&a_{2n}&b_2\\
    ...&...&...&...&...\\
    a_{n1}&a_{n2}&...&a_{nn}&b_n
\end{array}\end{bmatrix}.
$$

Zapisanie układu równań w postaci macierzowej ma wiele zalet. Jesteśmy w stanie w szybki sposób sprawdzić, czy równanie ma jednoznaczne rozwiązanie przez sprawdzenie czy wyznacznik macierzy głównej nie jest zerowy, gdyż jeśli $AX=B$, to $A^{-1}AX=X=A^{-1}B$. Musi więc istnieć macierz odwrotna. Sprawia to również, że zapis układu jest bardziej czytelny oraz pozwala ułatwić operowanie na takim układzie równań za pomocą komputera.

\subsection{Eliminacja Gaussa}

Metoda eliminacji Gaussa jest algorytmem stosowanym do rozwiązywania układu równań. Polega ona na doprowadzeniu macierzy do postaci schodkowej, tzn. zawierającej niezerowe wartości tylko na głównej przekątnej. W algorytmie dozwolone są tylko operacje na wierszach i kolumnach, czyli dodawanie lub odejmowanie od wiersza (kolumny) wielokrotności innego wiersza (kolumny) oraz zamienianie kolejności dwóch wierszy (kolumn).

Alternatywnie, na kursie algebry liniowej poznaliśmy metodę na odwracanie macierzy za pomocą eliminacji Gaussa. Wtedy z lewej stronie wpisujemy oryginalną macierz, z prawej macierz identyczności i dokonując operacji wierszowych na całości staramy się doprowadzić lewą macierz do macierzy identyczności. Wtedy to co, powstanie z prawej strony będzie szukaną macierzą odwrotną. W poniższej pracy nie skorzystamy z tej wariacji na tematy metody eliminacji Gaussa.

\subsection{Rozkład $QR$}

Każdą macierz $A$ $m\times n$ o wyrazach rzeczywistych taka, że $rank(A)=n$, można zapisać jako $A=QR$, gdzie $R$ jest macierzą górnotrójkątną, a $Q$ ma kolumny ortogonalne. Ponieważ my będziemy rozważać macierze $A$ będące reprezentacją jednoznacznych układów równań, to interesują nas tylko $A\in GL_n(\R)$.

Zauważmy, że jeśli $A$ ma niezerowy wyznacznik, to $A$ nie może mieć liniowo zależnych kolumn. W takim razie, wektory $a_1,...,a_n$ odpowiadające kolumnom $A$ są bazą przestrzeni $\R^n$ jako maksymalny możliwy układ wektorów liniowo niezależnych. Możemy na ich podstawie stworzyć bazę ortonormalną $u_1,...,u_n$ przez proces Grama-Schmidta. Wtedy dla $k=1,..,n$
$$u_k=a_k-\sum\limits_{i=1}^{k-1}{\langle u_i,a_k\rangle\over \langle u_i,u_i\rangle}u_i.$$
Co więcej, dla dowolnego $a_k$ z oryginalnej bazy możemy go zapisać za pomocą kombinacji liniowej wektorów z bazy ortonormalnej:
\begin{align*}
    a_k&=\sum\limits_{i=1}^n c_iu_i=\sum\limits_{i=1}^nc_i\sum\limits_{j=1}^{i-1} [a_k-\sum\limits_{i=1}^{k-1}{\langle u_i,a_k\rangle\over \langle u_i,u_i\rangle}u_i]
\end{align*}
a ponieważ $a_1,..,a_n$ były wektorami lnz, to dla $i> k$ $c_i=0$. Niech $r_k$ to będzie wektor zawierający współczynniki $c_i$ dla wektora $a_k$:
$$r_k=\begin{bmatrix}
    c_1\\
    c_2\\
    ...\\
    c_k\\
    0\\
    ...\\
    0
\end{bmatrix}$$
Czyli mamy, że
$$a_k=\begin{bmatrix}
    u_1&u_2&...&u_n
\end{bmatrix}r_k
$$
i dalej
$$
A=\begin{bmatrix}
    u_1&u_2&...&u_n
\end{bmatrix}\begin{bmatrix}
    r_1&r_2&...&r_n
\end{bmatrix}.
$$
Zauważamy, że $R=\begin{bmatrix}
    r_1&r_2&...&r_n
\end{bmatrix}$ to macierz górnotrójkątna, a $Q$ to macierz ortogonalna.

Niech teraz $A$ to macierz główna rozważanego układu równań, $Q,R$ to macierze z jej rozkładu, $X$ niech będzie wektorem wartości szukanych, a $B$ niech będzie wektorem wyrazów wolnych. Wtedy
\begin{align*}
    AX&=B\\
    (QR)X&=B
\end{align*}
i ponieważ dla macierzy ortonormalnych mamy $Q^{-1}=Q^T$, to w prosty sposób możemy zamienić powyższy układ na
$$RX=Q^TB.$$