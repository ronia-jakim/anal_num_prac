\documentclass[11pt, wide, leqno]{mwart}

\usepackage{../template}

\tit{P.0.12.}

\begin{document}
\maketitle
\tableofcontents

\section{Wstęp}

W matematyce bardzo często pojawiają się wartości niewymierne, takie jak $\ln 3$, których nie możemy wyrazić w sposób przystępny dla człowieka. Z tego powodu, powstało wiele metod przybliżania funkcji w określonych punktach. Jedną z nich jest użycie szeregu Taylora, opisanego przez Brooka Taylora w 1715 roku oraz wspomniana przez Jamesa Gregory'a w 1671 r.
\medskip

W swojej istocie twierdzenie Taylora mówi, że jeśli dana jest funkcja $f$ klasy $C^n$, czyli różniczkowalna $n$ razy w każdym punkcie jej dziedziny, to możemy ją przybliżyć w otoczeniu dowolnego punktu $a$ za pomocą szeregu:
$$f(x)=\sum\limits_{k=0}^n\big({(x-a)^k\over k!} f^{(k)}(a)\big)+R_n(x, a),$$
gdzie $R_n(x, a)$ spełnia
$$\lim\limits_{x\to a}{R_n(x,a)\over\|x-a\|^n}=0.$$
% nie chce tutaj indentu
Jak nietrudno zauważyć, wartość $R_n(x, a)$ przy $x$ bardzo blisko $a$ jest zaniedbywalnie mała, więc w trakcie obliczeń możemy ją pominąć.

Innym ciekawym zastosowaniem szeregów Taylora jest tworzenie grafów funkcji. Możemy zdefiniować na przykład $\tan x$ w pobliżu punktu $0$ jako zwykły wielomian, co pozwala na obliczenie przybliżonych wartości funkcji i naniesienie ich na oś współrzędnych. W matematyce szereg Taylora, który aproksymuje funkcje w pobliżu punktu $0$ nosi nazwę szeregu Maclaurina.

Celem niniejszego sprawozdania jest sprawdzenie dokładności przybliżania funkcji za pomocą szeregów Taylora. W \$\$2-3 omówiono dwa ważne zastosowania szeregów Taylora: przybliżanie wartości niewymiernych oraz rysowanie grafów funkcji. Dokładniej, w \$\$2. przedstawiono wyniki przybliżania wartości $\ln \frac12$ przy pomocy szeregów Maclaurina różnego stopnia. \$\$3. zajmuje się natomiast rysowaniem grafu funkcji $\tan x$ w pobliżu $0$ oraz porównaniem tego z wartościami otrzymanymi za pomocą bibliotecznej funkcji $\tan$ w języku Julia.

\section{Przybliżanie wartości $\ln\frac12$}

\subsection{Metoda}

W celu obliczenia wartości $\ln\frac12$ użyte zostanie przybliżanie funkcji
$$f(x)=\ln (1+x)$$
w pobliżu $a=0$ za pomocą szerega Maclaurina.

Wzór na pochodną funkcji $\ln (x+1)$ jest powszechnie znany:
$${d\over dx}\ln x=\frac1{x+1},$$
natomiast wzór na pochodną $k$-tego stopnia, można wyliczyć w prosty sposób:
\begin{align}
    {d^k\over dx^k}\ln (x+1)=(-1)^{k+1}{(k-1)!\over(1+x)^{k}}.
\end{align}

Wzór na szereg Taylora $\ln (1+x)$ w pobliżu $a=0$, to:
$$f(0)=\sum\limits_{i=0}^n{f^{(k)}(x)\over k!}(x-0)^n + R(x, 0)=\sum\limits_{i=1}^n(-1)^{i+1}{x^i\over i}+R(x, 0).$$



\koniec
\end{document}