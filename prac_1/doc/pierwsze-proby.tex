\section{Pierwsze próby}


\subsection{Interpretacja geometryczna}

Bardzo często $\pi$ jest definiowane jako stosunek obwodu okręgu do jego średnicy. W historii pojawiało się wiele prób wyznaczenia $\pi$ korzystając z obwodu wielokątów foremnych wpisanych w oraz opisanych na okręgu jednostkowym. Wraz ze wzrostem liczby boków zwiększa się dokładność oszacowań obwodu okręgu, co daje coraz to bliższe prawdy granice na wartość ludolfiny. 

Takie podejście stosował już w starożytności Archimedes. Wyprowadził on wzór rekursyjny na obwód $2n$-kąta foremnego wpisanego oraz opisanego na okręgu na podstawie obwodu $n$-kąta.

\begin{figure}[h]\centering
\begin{tikzpicture}
    \coordinate[label=left:$A_1$] (A1) at (-3, -3);
    \coordinate[label=left:$A_2$] (A2) at (-3, 3);
    \coordinate[label=right:$A_3$] (A3) at (3, 3);
    \coordinate[label=right:$A_4$] (A4) at (3, -3);

    \draw (A1)--(A2)--(A3)--(A4)--cycle;

    \coordinate[label=below:$A_1'$] (AA1) at (0, -3);
    \coordinate[label=left:$A_2'$] (AA2) at (-3, 0);
    \coordinate[label=above:$A_3'$] (AA3) at (0, 3);
    \coordinate[label=right:$A_4'$] (AA4) at (3, 0);

    \draw (AA1)--(AA2)--(AA3)--(AA4)--cycle;

    \coordinate[label=below:$B_1$] (B1) at (-1.22, -3);
    \coordinate[label=left:$B_2$] (B2) at (-3, -1.22);
    \coordinate[label=left:$B_3$] (B3) at (-3, 1.22);
    \coordinate[label=above:$B_4$] (B4) at (-1.22, 3);
    \coordinate[label=above:$B_5$] (B5) at (1.22, 3);
    \coordinate[label=right:$B_6$] (B6) at (3, 1.22);
    \coordinate[label=right:$B_7$] (B7) at (3, -1.22);
    \coordinate[label=below:$B_8$] (B8) at (1.22, -3);

    \draw (B1)--(B2)--(B3)--(B4)--(B5)--(B6)--(B7)--(B8)--cycle;
    
    \draw (3, 0)--(0, 0);
    \node at (1.5, 0.3) {$r=1$};

    \draw[very thick, ziel] (0, 0) circle (3);
    
\end{tikzpicture}
\caption{Wielokąty opisane i wpisane w okrąg o promieniu $1$.}
\label{pierwszy}
\end{figure}

Wpiszmy $n$-kąt foremny w okrąg o promieniu $1$. Teraz na tym samym okręgu opiszmy $n$-kąt tak, żeby wierzchołki wielokąta wpisanego były srodkami boków wielokąta opisywaneg. Dostajemy w ten sposob $n$-kąt foremny opisany na okręgu o promieniu $1$. Nietrudno zauważyć, że teraz jeśli połączymy sąsiednie boki $n$-kąta opisanego odcinkami styczymi do okręgu o końcach w równej odległości od najbliższego wierzchołka, to dostaniemy $2n$-kąt foremny. Sytuacja dla $n=4$ została  przedstawiona na Rysunku~\ref{pierwszy}.

Rozważmy teraz trójkąt $\Delta A_1'A_1A_2$. Zawuażmy, że odcinek $\overline{B_1B_2}$ dzieli go na dwa trójkąty podobne:
$$\Delta B_1A_1B_2\sim\Delta A_1'A_1A_2'.$$
Dla przejżystości zapisów oznaczmy $|\overline{A_1A_2}|=A$, $|\overline{B_1B_2}|=B$ oraz $|\overline{A_1'A_2'}|=a$. Z proporcji w trójkątach podobnych mamy:
\begin{align*}
    {B\over \frac12A-\frac12B}&={a\over \frac 12A}\\
    B&=\frac aA(A-B)\\
    B&=a-\frac aAB\\
    B&={aA\over A+a}
\end{align*}
Oznaczmy teraz obwód $n$-kąta wpisanego jako $l_n$, a $n$-kąta opisanego - $L_n$. Według Rysunku~\ref{pierwszy} są one równe:
\begin{align*}
    l_n&=na\\
    L_n&=nA\\
    L_{2n}&=2nB=2n{aA\over A+a}=2n^2{aA\over An+an}=2{L_nl_n\over L_n+l_n}
\end{align*}

Dalej, oznaczmy długość boku $2n$-kąta wpisanego jako $b$. Zauważmy, że wówczas:
\begin{align*}
    B&=2\tan{\pi\over 2n}\\
    a&=2\sin{\pi\over n}\\
    b&=2\sin{\pi\over 2n}
\end{align*}
oraz:
\begin{align*}
    l_{2n}&=2nb=4n\sin{\pi\over 2n}=\sqrt{16n^2\sin^2{\pi\over 2n}}=\sqrt{8n^2{\sin{\pi\over2n}\over\cos{\pi\over 2n}}2\sin{\pi\over 2n}\cos{\pi\over 2n}}=\\
    &=\sqrt{8n^2\tan{\pi\over 2n}\sin{\pi\over n}}=\sqrt{2nBna}=\sqrt{L_{2n}l_n}.
\end{align*}


\subsection{Algorytm Monte Carlo}

Tak jak w poprzedniej metodzie, możemy skorzystać z faktu, że dla koła jednostkowego $\pi$ jest równe jego polu. Zauważmy, że jeżeli będziemy wybierać losowo punkty kwadratu o polu 1, to $\frac\pi4$ z nich powinno znaleźć się w ćwiartce koła o środku w jednym z wierzchołków tego kwadratu (~Rysunek \ref{fig:monte-carlo}.).

\begin{figure}[!h]\centering
\begin{tikzpicture}
    \coordinate (A) at (0,0);
    \coordinate (B) at (4,0);
    \coordinate (C) at (4,4);
    \coordinate (D) at (0,4);

    \filldraw[fill=ziel!40!white, draw=ziel] (B) arc[start angle=0, end angle=90, radius=4];
    \filldraw[color=ziel!40!white] (A)--(B)--(D)--cycle;
    \draw[thick] (A)--(B)--(C)--(D)--cycle;

    \node at (1.5, 2) {\Large$\frac\pi4$};
    \node at (2, -0.3) {\large$1$};
    \node at (-0.3, 2) {\large$1$};
\end{tikzpicture}
\caption{Stosunek pola ćwiartki koła jednostkowego do kwadratu o boku $1$}
\label{fig:monte-carlo}
\end{figure}

Korzystając z algorytmu Monte Carlo możemy wybierać losowo współrzędne $x,y\in[0,1]$ kolejnych punktów, a następnie sprawdzać ile z nich spełnia warunek
$$x^2+y^2\leq1.$$
Otrzymany stosunek będzie coraz bliższy $\frac\pi4$ wraz ze zwiększaniem ilości testowanych punktów.

{\color{cyan}NAKLEPAĆ I TYM LOGIEM PRZYBLIŻYĆ ZBIEŻNOŚĆ CZY INNE CHUJU MUJU}


\subsection{Szereg Taylora}
W matematyce bardzo często w celu przybliżania porządanych wartości używa się szeregów Taylora. Tak dla przykładu, korzystając z rozszerzenia funkcji $\arctan x$ w punkcie $0$ możemy oszacować wartość $\frac\pi4$:
\begin{equation}
\begin{split}
    \frac\pi4&=\arctan1=\sum\limits_{k=0}^\infty{\arctan^{(k)}0\over k!}(1-0)^k=\\
    &=1-\frac13+\frac15-\frac17+...=\sum\limits_{k=0}^\infty{(-1)^k\over 2k+1}.
\end{split}
\end{equation}

W obliczeniach praktycznych nie możliwe jest dodawanie kolejnych elementów sumy w nieszkończoność. Konieczne jest więc zatrzymanie się na pewnym $N$, co daje pewien błąd, $R_N$:
$$\frac\pi4\approx \sum\limits_{k=0}^N{(-1)^k\over 2k+1}+R_N.$$
Oznaczmy tę sumę jako $P_N$. Ponieważ dla przybliżeń funkcji szeregiem Taylora coraz wyższego stopnia dostajemy coraz dokładniejszy wynik, to $P_{N+1}$ powinno być dokładniejsze niż $P_N$. Zauważamy też, że
$$P_{N+1}-P_N={(-1)^{N+1}\over 2N+3}$$
w takim razie możemy oszacować błąd dla szeregu Taylora $N$-tego stopnia za pomocą
$$R_N\approx \max{(-1)^{N+1}\over 2N+3},$$
co daje zbieżność liniową.

Problem tego przybliżenia $\pi$ został przeanalizowany już przez Madhawa z Sangamagramy w XIV wieku. Zaproponował on następującą korekcję wzoru dla skończonych sum:
\begin{equation}
    \frac\pi4\approx \sum\limits_{k=0}^N{(-1)^k\over 2k+1}\pm{N^2+1\over 4N^3+5N}.
\end{equation}

{\color{cyan}WYPADAŁOBY NAKLEPAĆ I PRZEDSTAWIĆ WYNIKI}


