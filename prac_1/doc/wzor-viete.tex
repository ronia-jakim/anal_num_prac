\section{Wzór Viete'a}

% https://en.wikipedia.org/wiki/Vi%C3%A8te%27s_formula

Viete wyprowadził swoją formułę na $\pi$ obserwując stosunek pola $2^n$-kata foremnego do pola $2^{n+1}$-kata foremnego. Poprzez zwiększanie $n$ w nieskończoność, jesteśmy w stanie dostać stosunek $2^2$-kąta foremnego, czyli kwadratu, do pola koła w które został on wpisany. Można ją też wyprowadzić za pomocą tożsamości udowodnionej przez Eulera ponad 100 lat po śmierci Viete'a.

Wzór zaproponowany przez Viete'a, uznawany za prekursor analizy matematycznej w matematyce poprzez pierwsze wykorzystanie nieskończonego ilorazu, jest następujący:
\begin{equation}
    {2\over\pi}=\prod\limits_{k=1}^n{a_k\over2},
\end{equation}
gdzie $a_1=\sqrt2$ oraz
$$a_k=\sqrt{2+a_{n-1}}.$$

Wiemy, że
$$\sin{x}=2\sin{\frac x2}\cos{\frac x2}$$
co dla pewnego $n$ pozwala wyrazić się jako
$$\sin{x}=2^n\sin{{x\over 2^n}}\prod\limits_{k=1}^n\cos{x\over 2^k}$$
Weźmy $x={\pi\over 2}$. Wiemy, że $\sin{\frac\pi2}=1$ oraz




$$\pi=\lim\limits_{k\to\infty}2^k\sqrt{2-a_k}$$
$$a_1=0$$
$$a_k=\sqrt{2+a_{k+1}}$$